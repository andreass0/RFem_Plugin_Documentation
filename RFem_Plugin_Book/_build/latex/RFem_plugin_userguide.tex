%% Generated by Sphinx.
\def\sphinxdocclass{jupyterBook}
\documentclass[letterpaper,10pt,english]{jupyterBook}
\ifdefined\pdfpxdimen
   \let\sphinxpxdimen\pdfpxdimen\else\newdimen\sphinxpxdimen
\fi \sphinxpxdimen=.75bp\relax
\ifdefined\pdfimageresolution
    \pdfimageresolution= \numexpr \dimexpr1in\relax/\sphinxpxdimen\relax
\fi
%% let collapsible pdf bookmarks panel have high depth per default
\PassOptionsToPackage{bookmarksdepth=5}{hyperref}
%% turn off hyperref patch of \index as sphinx.xdy xindy module takes care of
%% suitable \hyperpage mark-up, working around hyperref-xindy incompatibility
\PassOptionsToPackage{hyperindex=false}{hyperref}
%% memoir class requires extra handling
\makeatletter\@ifclassloaded{memoir}
{\ifdefined\memhyperindexfalse\memhyperindexfalse\fi}{}\makeatother

\PassOptionsToPackage{warn}{textcomp}

\catcode`^^^^00a0\active\protected\def^^^^00a0{\leavevmode\nobreak\ }
\usepackage{cmap}
\usepackage{fontspec}
\defaultfontfeatures[\rmfamily,\sffamily,\ttfamily]{}
\usepackage{amsmath,amssymb,amstext}
\usepackage{polyglossia}
\setmainlanguage{english}



\setmainfont{FreeSerif}[
  Extension      = .otf,
  UprightFont    = *,
  ItalicFont     = *Italic,
  BoldFont       = *Bold,
  BoldItalicFont = *BoldItalic
]
\setsansfont{FreeSans}[
  Extension      = .otf,
  UprightFont    = *,
  ItalicFont     = *Oblique,
  BoldFont       = *Bold,
  BoldItalicFont = *BoldOblique,
]
\setmonofont{FreeMono}[
  Extension      = .otf,
  UprightFont    = *,
  ItalicFont     = *Oblique,
  BoldFont       = *Bold,
  BoldItalicFont = *BoldOblique,
]



\usepackage[Bjarne]{fncychap}
\usepackage[,numfigreset=1,mathnumfig]{sphinx}

\fvset{fontsize=\small}
\usepackage{geometry}


% Include hyperref last.
\usepackage{hyperref}
% Fix anchor placement for figures with captions.
\usepackage{hypcap}% it must be loaded after hyperref.
% Set up styles of URL: it should be placed after hyperref.
\urlstyle{same}

\addto\captionsenglish{\renewcommand{\contentsname}{Inhalt}}

\usepackage{sphinxmessages}



        % Start of preamble defined in sphinx-jupyterbook-latex %
         \usepackage[Latin,Greek]{ucharclasses}
        \usepackage{unicode-math}
        % fixing title of the toc
        \addto\captionsenglish{\renewcommand{\contentsname}{Contents}}
        \hypersetup{
            pdfencoding=auto,
            psdextra
        }
        % End of preamble defined in sphinx-jupyterbook-latex %
        

\title{RFEM-Plugin for SIMULTAN (Work in Progress)}
\date{Aug 26, 2022}
\release{}
\author{Copyright (c) 2022 RWT Plus ZT GmbH and TU Wien}
\newcommand{\sphinxlogo}{\vbox{}}
\renewcommand{\releasename}{}
\makeindex
\begin{document}

\pagestyle{empty}
\sphinxmaketitle
\pagestyle{plain}
\sphinxtableofcontents
\pagestyle{normal}
\phantomsection\label{\detokenize{intro::doc}}


\sphinxAtStartPar
This user guide should help navigate the RFEM\sphinxhyphen{}Plugin which enables structural analysis with the SIMULTAN datamodel based
on the finite element
software \sphinxhref{https://www.dlubal.com/en/downloads-and-information/documents/online-manuals/rfem-6}{RFEM 6 by Dlubal}.
\begin{itemize}
\item {} 
\sphinxAtStartPar
Inhalt

\begin{itemize}
\item {} 
\sphinxAtStartPar
{\hyperref[\detokenize{Introduction::doc}]{\sphinxcrossref{Introduction}}}

\item {} 
\sphinxAtStartPar
{\hyperref[\detokenize{Getting_started_with_the_RFem_Plugin::doc}]{\sphinxcrossref{Getting started with the RFEM Plugin}}}

\item {} 
\sphinxAtStartPar
{\hyperref[\detokenize{Setting_up_a_problem::doc}]{\sphinxcrossref{Setting up a problem}}}

\item {} 
\sphinxAtStartPar
{\hyperref[\detokenize{Running_a_simulation::doc}]{\sphinxcrossref{Running a simulation}}}

\item {} 
\sphinxAtStartPar
{\hyperref[\detokenize{Results_of_the_simulation::doc}]{\sphinxcrossref{Results of the simulation}}}

\item {} 
\sphinxAtStartPar
{\hyperref[\detokenize{SIMULTAN_Datastructure_to_incorporate_the_RFem_Data_model::doc}]{\sphinxcrossref{SIMULTAN Datastructure to incorporate the RFEM Data model}}}

\item {} 
\sphinxAtStartPar
{\hyperref[\detokenize{LICENSE::doc}]{\sphinxcrossref{Copyright and license agreements}}}

\item {} 
\sphinxAtStartPar
{\hyperref[\detokenize{References::doc}]{\sphinxcrossref{References}}}

\end{itemize}
\end{itemize}

\sphinxstepscope


\chapter{Introduction}
\label{\detokenize{Introduction:introduction}}\label{\detokenize{Introduction:id1}}\label{\detokenize{Introduction::doc}}

\section{What is the RFEM\sphinxhyphen{}Plugin?}
\label{\detokenize{Introduction:what-is-the-rfem-plugin}}
\sphinxAtStartPar
The RFEM\sphinxhyphen{}plugin (in the following referred to as “the plugin”) enables structural analysis with the finite element
software \sphinxhref{https://www.dlubal.com/en/downloads-and-information/documents/online-manuals/rfem-6}{RFEM 6 by Dlubal} for the
SIMULTAN\sphinxhyphen{}data model. This is achieved by exporting all necessary data for rudimental structural analysis into an
XML\sphinxhyphen{}file which can be imported into RFEM 6.


\section{Functionality covered by the RFEM\sphinxhyphen{}Plugin}
\label{\detokenize{Introduction:functionality-covered-by-the-rfem-plugin}}
\sphinxAtStartPar
In the current state of development the plugin is able to export:
\begin{itemize}
\item {} 
\sphinxAtStartPar
Geometry

\item {} 
\sphinxAtStartPar
Material data

\item {} 
\sphinxAtStartPar
Loads
\begin{itemize}
\item {} 
\sphinxAtStartPar
Nodal

\item {} 
\sphinxAtStartPar
Line

\item {} 
\sphinxAtStartPar
Surface

\end{itemize}

\item {} 
\sphinxAtStartPar
Supports with certain nonlinear behaviour
\begin{itemize}
\item {} 
\sphinxAtStartPar
Nodal

\item {} 
\sphinxAtStartPar
Line

\item {} 
\sphinxAtStartPar
Surface

\end{itemize}

\item {} 
\sphinxAtStartPar
Structural components and their hinges with certain nonlinear behaviour
\begin{itemize}
\item {} 
\sphinxAtStartPar
Beams

\item {} 
\sphinxAtStartPar
Columns

\item {} 
\sphinxAtStartPar
Surfaces

\end{itemize}

\end{itemize}

\sphinxAtStartPar
What we are not able to do right now:
\begin{itemize}
\item {} 
\sphinxAtStartPar
Import results back into the SIMULTAN\sphinxhyphen{}data model (since in its current state there is now mesh visualization
implemented in SIMULTAN\sphinxhyphen{}data model)

\item {} 
\sphinxAtStartPar
Track changes made in the RFEM\sphinxhyphen{}data model but not in the SIMULTAN\sphinxhyphen{}data model

\end{itemize}


\section{Project}
\label{\detokenize{Introduction:project}}
\sphinxAtStartPar
The development of the RFEM\sphinxhyphen{}Plugin is the result of the FFG\sphinxhyphen{}research project “Ganzheitliche Gebaeudesimulation”
supervised by Christoph Bauer. Additional information about the project and project outcomes can be found under these
links:
\begin{itemize}
\item {} 
\sphinxAtStartPar
\sphinxhref{https://www.woschitzgroup.com/news/bim-war-gestern/}{Ganzheitliche Gebaeudesimulation Press}

\item {} 
\sphinxAtStartPar
\DUrole{xref,myst}{RFEM Plugin Publication}

\item {} 
\sphinxAtStartPar
\DUrole{xref,myst}{IDA\sphinxhyphen{}ICE Plugin Publication}

\item {} 
\sphinxAtStartPar
\DUrole{xref,myst}{Additional works}

\end{itemize}

\sphinxAtStartPar
Information and publications about SIMULTAN:
\begin{itemize}
\item {} 
\sphinxAtStartPar
\sphinxhref{https://repositum.tuwien.at/handle/20.500.12708/62532}{SIMULTAN as a Big\sphinxhyphen{}Open\sphinxhyphen{}Real\sphinxhyphen{}BIM Data Model \sphinxhyphen{} Proof of Concept for the Design Phase}
{[}\hyperlink{cite.References:id8}{PLW+19}{]}

\item {} 
\sphinxAtStartPar
\DUrole{xref,myst}{Digital Twin applications using the SIMULTAN data model and Python} {[}\hyperlink{cite.References:id5}{BuhlerSB22}{]}

\item {} 
\sphinxAtStartPar
SIMULTAN as a Big\sphinxhyphen{}Open\sphinxhyphen{}Real\sphinxhyphen{}BIM Data Model \sphinxhyphen{} Proof of Concept for the Design Phase
{[}\hyperlink{cite.References:id8}{PLW+19}{]}

\item {} 
\sphinxAtStartPar
SIMULTAN \sphinxhyphen{} Simultane Planungsumgebung fuer Gebaeudecluster in resilienten, ressourcen\sphinxhyphen{} und hoechst energieeffizienten
Stadtteilen {[}\hyperlink{cite.References:id4}{Betal}{]}

\end{itemize}


\section{Authors}
\label{\detokenize{Introduction:authors}}
\sphinxAtStartPar
Project supervision of the development of the RFEM\sphinxhyphen{}Plugin was done by Andreas Sarkany and Bernhard Steiner. Core
development of the software was conducted by Zsombor Jarosi. Additional Feedback and testing was provided by Thomas Rabl
and Thomas Bednar.


\section{Getting help}
\label{\detokenize{Introduction:getting-help}}
\sphinxAtStartPar
You can get in touch with the authors via the Issue\sphinxhyphen{}Tracker of the repository this user guide is published on.

\sphinxstepscope


\chapter{Getting started with the RFEM Plugin}
\label{\detokenize{Getting_started_with_the_RFem_Plugin:getting-started-with-the-rfem-plugin}}\label{\detokenize{Getting_started_with_the_RFem_Plugin:id1}}\label{\detokenize{Getting_started_with_the_RFem_Plugin::doc}}

\section{Installing the SIMULTAN\sphinxhyphen{}Editor}
\label{\detokenize{Getting_started_with_the_RFem_Plugin:installing-the-simultan-editor}}
\sphinxAtStartPar
To use the plugin you need to install the latest version of the SIMULTAN Editor, which can be
found \sphinxhref{https://github.com/bph-tuwien/SIMULTAN.Documentation/wiki}{here}. The SIMULTAN Editor enables you to interact
with the SIMULTAN data model through a user interface. A comprehensive user\sphinxhyphen{}guide how to install and interact with the
SIMULTAN Editor can be found under this \sphinxhref{https://github.com/bph-tuwien/SIMULTAN.Documentation/wiki}{link}.


\section{Installing the RFEM Plugin}
\label{\detokenize{Getting_started_with_the_RFem_Plugin:installing-the-rfem-plugin}}
\sphinxAtStartPar
After installing the SIMULTAN Editor you can install the RFEM Plugin via the Plugin Manager \hyperref[\detokenize{Getting_started_with_the_RFem_Plugin:plugin-manager}]{Fig.\@ \ref{\detokenize{Getting_started_with_the_RFem_Plugin:plugin-manager}}}.
Either download the plugin from the SIMULTAN Server (not possible yet) or install from your local disk with the
installation file.

\begin{figure}[htbp]
\centering
\capstart

\noindent\sphinxincludegraphics[height=250\sphinxpxdimen]{{plugin_manager}.png}
\caption{Plugin Manager in he SIMULTAN Editor.}\label{\detokenize{Getting_started_with_the_RFem_Plugin:plugin-manager}}\end{figure}

\sphinxAtStartPar
When the installation was successful a new tab will be added to your taskbar named RFEM Plugin
\hyperref[\detokenize{Getting_started_with_the_RFem_Plugin:rfem-plugin-new-tab}]{Fig.\@ \ref{\detokenize{Getting_started_with_the_RFem_Plugin:rfem-plugin-new-tab}}}.

\begin{figure}[htbp]
\centering
\capstart

\noindent\sphinxincludegraphics[height=200\sphinxpxdimen]{{rfem_plugin_new_tab}.png}
\caption{New tab added to the taskbar of the SIMULTAN Editor after installing the RFEM Plugin.}\label{\detokenize{Getting_started_with_the_RFem_Plugin:rfem-plugin-new-tab}}\end{figure}


\section{Run the example}
\label{\detokenize{Getting_started_with_the_RFem_Plugin:run-the-example}}
\sphinxAtStartPar
To check if everything was installed properly please run one of the examples provided where
you \sphinxhref{https://github.com/bph-tuwien/GBS.Plugins/releases}{downloaded} the installation file of the plugin. Open the
example in your SIMULTAN Editor (username and password is “admin”), and follow the instructions in the
chapter {\hyperref[\detokenize{Running_a_simulation:running-a-simulation}]{\sphinxcrossref{\DUrole{std,std-ref}{Running a simulation}}}}. If this works fine and you are getting plausible results the
installation was successful.

\sphinxstepscope


\chapter{Setting up a problem}
\label{\detokenize{Setting_up_a_problem:setting-up-a-problem}}\label{\detokenize{Setting_up_a_problem::doc}}
\sphinxAtStartPar
A short overview of the representation of the data in the data model will be given in the beginning. After that the
tutorial focuses on modelling with the implemented user interface of the plugin.


\section{Structural analysis component}
\label{\detokenize{Setting_up_a_problem:structural-analysis-component}}
\sphinxAtStartPar
In \hyperref[\detokenize{Setting_up_a_problem:struct-comp}]{Fig.\@ \ref{\detokenize{Setting_up_a_problem:struct-comp}}} a simplified depiction of the structural analysis component is given. This component is used in
the SIMULTAN datamodel to store the information needed to export data with the RFEM Plugin to enable structural analysis
simulations in RFEM 6.

\begin{figure}[htbp]
\centering
\capstart

\noindent\sphinxincludegraphics[height=250\sphinxpxdimen]{{struct_component}.jpg}
\caption{Simplified depiction of the representation of the structural analysis component.}\label{\detokenize{Setting_up_a_problem:struct-comp}}\end{figure}

\sphinxAtStartPar
A more in depth look into the SIMULTAN representation of the structural analysis data is given in the
chapter {\hyperref[\detokenize{SIMULTAN_Datastructure_to_incorporate_the_RFem_Data_model::doc}]{\sphinxcrossref{\DUrole{doc,std,std-doc}{SIMULTAN Datastructure to incorporate the RFEM Data model}}}}
.

\begin{sphinxadmonition}{warning}{Warning:}
\sphinxAtStartPar
Although this data structure can be implemented manually, it is strongly recommended to use the user interface that
comes with the plugin for modelling. Manuel modelling is always prone to errors and could result in very time
consuming efforts when trying to fix those.
\end{sphinxadmonition}


\section{Geometrical modelling}
\label{\detokenize{Setting_up_a_problem:geometrical-modelling}}
\sphinxAtStartPar
The geometry is created in the Geometry Editor of the SIMULTAN Editor. Please consult
the \sphinxhref{https://github.com/bph-tuwien/SIMULTAN.Documentation/wiki}{SIMULTAN Editor User Guide}
for further information on how to use this Geometry Editor.


\subsection{Modelling guidelines}
\label{\detokenize{Setting_up_a_problem:modelling-guidelines}}
\sphinxAtStartPar
In \hyperref[\detokenize{Setting_up_a_problem:simultan-geometry}]{Fig.\@ \ref{\detokenize{Setting_up_a_problem:simultan-geometry}}} it can be seen that the plugin uses a reference geometry (white) as well as a geometry
for the structural analysis (red). The reference geometry, representing the outer building envelope, can for example be
an architectural model or the model used for the building physics simulations.

\begin{figure}[htbp]
\centering
\capstart

\noindent\sphinxincludegraphics[height=350\sphinxpxdimen]{{simultan_geometry}.png}
\caption{Example of the geometry of a \sphinxstyleemphasis{Tiny House} modelled with the Geometry Editor of the SIMULTAN Editor.}\label{\detokenize{Setting_up_a_problem:simultan-geometry}}\end{figure}

\begin{sphinxadmonition}{tip}{Tip:}
\sphinxAtStartPar
The structural model (red) can be easily created by copying the surfaces of the reference geometry to the middle axis
of the structural building components.
\end{sphinxadmonition}

\sphinxAtStartPar
For the simulation the structural geometry is exported to RFEM, therefore all of information exclusively used for
structural analysis will be linked to this geometry.


\section{User Interface}
\label{\detokenize{Setting_up_a_problem:user-interface}}
\sphinxAtStartPar
The RFEM\sphinxhyphen{}Plugin comes with a user interface \hyperref[\detokenize{Setting_up_a_problem:ui-plugin}]{Fig.\@ \ref{\detokenize{Setting_up_a_problem:ui-plugin}}} to help with the creation of the needed data structure
for the plugin. The user interface itself is closely built around the graphical interfaces of RFEM 6. With this we hope
to provide a familiar experience to all the RFEM users when using this plugin.

\begin{figure}[htbp]
\centering
\capstart

\noindent\sphinxincludegraphics[height=250\sphinxpxdimen]{{ui_plugin}.jpg}
\caption{Example of the user interface to create the data structure for the plugin when adding a new material.}\label{\detokenize{Setting_up_a_problem:ui-plugin}}\end{figure}


\section{Connecting components with the geometry}
\label{\detokenize{Setting_up_a_problem:connecting-components-with-the-geometry}}
\sphinxAtStartPar
The before created components are a representation of the information needed for the structural analysis. Connect the
components to the geometry as needed. Please consult
the \sphinxhref{https://github.com/bph-tuwien/SIMULTAN.Documentation/wiki}{SIMULTAN Editor User Guide}
for general information on how to link components and geometrical information.

\sphinxAtStartPar
An outline of the most important links and how to create them will be given here.


\subsection{Surfaces}
\label{\detokenize{Setting_up_a_problem:surfaces}}
\sphinxAtStartPar
Information for surfaces are autogenerated via the already existing reference geometry since this data is not exclusive
to structural analysis.


\subsection{Members}
\label{\detokenize{Setting_up_a_problem:members}}
\sphinxAtStartPar
Geometrical lines connected to physical properties for structural analysis, e.g. beams or columns, are called members in
RFEM. To add a \sphinxcode{\sphinxupquote{Section\sphinxhyphen{}Component}} existing in the data model to a line follow these steps:
\begin{itemize}
\item {} 
\sphinxAtStartPar
select the line in the \sphinxcode{\sphinxupquote{Geometry Editor}}

\item {} 
\sphinxAtStartPar
Open the \sphinxcode{\sphinxupquote{Property Editor}} for the selected line

\item {} 
\sphinxAtStartPar
Choose the Component you want to link from the pop\sphinxhyphen{}up window

\end{itemize}

\begin{figure}[htbp]
\centering
\capstart

\noindent\sphinxincludegraphics[height=350\sphinxpxdimen]{{add_member}.png}
\caption{Example how to link a \sphinxcode{\sphinxupquote{Section\sphinxhyphen{}Component}} to a geometrical line in the \sphinxcode{\sphinxupquote{Geometry Editor}} using its \sphinxcode{\sphinxupquote{Property Editor}}.}\label{\detokenize{Setting_up_a_problem:add-member}}\end{figure}


\subsection{Supports}
\label{\detokenize{Setting_up_a_problem:supports}}
\sphinxAtStartPar
Supports are linked to vertices/lines/surfaces in the geometrical model. To add an existing \sphinxcode{\sphinxupquote{Support\sphinxhyphen{}Component}} to a
geometry in the geometrical model follow these steps:
\begin{itemize}
\item {} 
\sphinxAtStartPar
Select the node/line/surface in the \sphinxcode{\sphinxupquote{Geometry Editor}}

\item {} 
\sphinxAtStartPar
Open the \sphinxcode{\sphinxupquote{Property Editor}} for the selected node

\item {} 
\sphinxAtStartPar
Choose the Component you want to link from the pop\sphinxhyphen{}up window

\end{itemize}

\begin{figure}[htbp]
\centering
\capstart

\noindent\sphinxincludegraphics[height=350\sphinxpxdimen]{{add_support}.png}
\caption{Example how to link a \sphinxcode{\sphinxupquote{Support\sphinxhyphen{}Component}} to a geometrical node in the \sphinxcode{\sphinxupquote{Geometry Editor}} using its \sphinxcode{\sphinxupquote{Property Editor}}.}\label{\detokenize{Setting_up_a_problem:add-support}}\end{figure}


\subsection{Loads}
\label{\detokenize{Setting_up_a_problem:loads}}
\sphinxAtStartPar
Loads are linked to vertices/lines/surfaces in the geometrical model. To add an existing \sphinxcode{\sphinxupquote{Load\sphinxhyphen{}Component}} to a geometry
in the geometrical model follow these steps:
\begin{itemize}
\item {} 
\sphinxAtStartPar
Select the node/line/surface in the \sphinxcode{\sphinxupquote{Geometry Editor}}

\item {} 
\sphinxAtStartPar
Open the \sphinxcode{\sphinxupquote{Property Editor}} for the selected node

\item {} 
\sphinxAtStartPar
Choose the Component you want to link from the pop\sphinxhyphen{}up window

\end{itemize}

\begin{figure}[htbp]
\centering
\capstart

\noindent\sphinxincludegraphics[height=350\sphinxpxdimen]{{add_load}.png}
\caption{Example how to link a \sphinxcode{\sphinxupquote{Load\sphinxhyphen{}Component}} to a geometrical node in the \sphinxcode{\sphinxupquote{Geometry Editor}} using its \sphinxcode{\sphinxupquote{Property Editor}}.}\label{\detokenize{Setting_up_a_problem:add-load}}\end{figure}

\begin{sphinxadmonition}{note}{Note:}
\sphinxAtStartPar
The same \sphinxcode{\sphinxupquote{Support\sphinxhyphen{}Component}} or \sphinxcode{\sphinxupquote{Load\sphinxhyphen{}Component}} can be linked to multiple different geometrical instances.
For example, if the same force is attacking on two differnet surfaces, the same component is used from the data model.
An in depth look at this you can find in {\hyperref[\detokenize{SIMULTAN_Datastructure_to_incorporate_the_RFem_Data_model::doc}]{\sphinxcrossref{\DUrole{doc,std,std-doc}{this chapter}}}}.
\end{sphinxadmonition}

\sphinxstepscope


\chapter{Running a simulation}
\label{\detokenize{Running_a_simulation:running-a-simulation}}\label{\detokenize{Running_a_simulation:id1}}\label{\detokenize{Running_a_simulation::doc}}

\section{Exporting the SIMULTAN\sphinxhyphen{}model to RFEM}
\label{\detokenize{Running_a_simulation:exporting-the-simultan-model-to-rfem}}
\sphinxAtStartPar
When modelling according to the conventions described in chapter {\hyperref[\detokenize{Setting_up_a_problem::doc}]{\sphinxcrossref{\DUrole{doc,std,std-doc}{Setting up a problem}}}} the
model is ready to be exported. This can be done by clicking the \sphinxcode{\sphinxupquote{Export Model}} button when you are inside the RFEM Plugin in
the SIMULTAN Editor \hyperref[\detokenize{Running_a_simulation:export}]{Fig.\@ \ref{\detokenize{Running_a_simulation:export}}}.

\begin{figure}[htbp]
\centering
\capstart

\noindent\sphinxincludegraphics[height=250\sphinxpxdimen]{{export}.png}
\caption{Export button to create the XML\sphinxhyphen{}file to import into RFEM 6.}\label{\detokenize{Running_a_simulation:export}}\end{figure}

\sphinxAtStartPar
After clicking the button you will be prompted with a \sphinxcode{\sphinxupquote{save dialogue}}. Please specify the location where you want to
save the XML\sphinxhyphen{}file and confirm.


\section{Importing into RFEM}
\label{\detokenize{Running_a_simulation:importing-into-rfem}}
\sphinxAtStartPar
To import the XML\sphinxhyphen{}file open RFEM 6 and choose the \sphinxcode{\sphinxupquote{Import XML function}} \hyperref[\detokenize{Running_a_simulation:import-xml}]{Fig.\@ \ref{\detokenize{Running_a_simulation:import-xml}}}.

\begin{figure}[htbp]
\centering
\capstart

\noindent\sphinxincludegraphics[height=250\sphinxpxdimen]{{import_xml}.png}
\caption{Import XML function in RFEM 6.}\label{\detokenize{Running_a_simulation:import-xml}}\end{figure}

\sphinxAtStartPar
Navigate to the location on your disk where you have saved the exported model and confirm to load it. After the model is
loaded the it will appear in RFEM 6 and is ready for further manipulation \hyperref[\detokenize{Running_a_simulation:model-in-rfem}]{Fig.\@ \ref{\detokenize{Running_a_simulation:model-in-rfem}}}.

\begin{figure}[htbp]
\centering
\capstart

\noindent\sphinxincludegraphics[height=250\sphinxpxdimen]{{model_in_rfem}.jpg}
\caption{Imported model in RFEM 6 based on an export from a SIMULTAN model.}\label{\detokenize{Running_a_simulation:model-in-rfem}}\end{figure}

\begin{sphinxadmonition}{warning}{Warning:}
\sphinxAtStartPar
In the current state of development the plugin is \sphinxstylestrong{NOT} bidirectional. Any adaptations done in RFEM are only
available in RFEM and cannot be retrieved into the SIMULTAN datamodel. This is a funcitonality which will be
implemented in the future. For now use RFEM as a solver and adjust the model in the SIMULTAN Editor.
\end{sphinxadmonition}


\section{Running a simulation in RFEM}
\label{\detokenize{Running_a_simulation:running-a-simulation-in-rfem}}
\sphinxAtStartPar
To run a simulation create load combinations in RFEM and run the simulation as usual in the finite element software.

\sphinxstepscope


\chapter{Results of the simulation}
\label{\detokenize{Results_of_the_simulation:results-of-the-simulation}}\label{\detokenize{Results_of_the_simulation:id1}}\label{\detokenize{Results_of_the_simulation::doc}}

\section{Visual results}
\label{\detokenize{Results_of_the_simulation:visual-results}}
\sphinxAtStartPar
For now the simulation results can only be viewed in RFEM itself, \hyperref[\detokenize{Results_of_the_simulation:rfem-sim-results}]{Fig.\@ \ref{\detokenize{Results_of_the_simulation:rfem-sim-results}}}.

\begin{figure}[htbp]
\centering
\capstart

\noindent\sphinxincludegraphics[height=350\sphinxpxdimen]{{rfem_sim_results}.png}
\caption{Results of a FE\sphinxhyphen{}analysis with RFEM 6 for an exported SIMULTAN model.}\label{\detokenize{Results_of_the_simulation:rfem-sim-results}}\end{figure}


\section{Additional information}
\label{\detokenize{Results_of_the_simulation:additional-information}}
\sphinxAtStartPar
This restriction is due to the fact that the SIMULTAN Editor has no functionality for mesh visualization implemented
yet. Since simulation results of finite element calculations are dependent on proper visualization it was decided to
delay this feature till the SIMULTAN Editor can provide this functionality.

\begin{sphinxadmonition}{note}{Note:}
\sphinxAtStartPar
As a workaround one can attach the calculation protocol and the RFEM model to the \sphinxcode{\sphinxupquote{Structural Analysis Component}}.
\end{sphinxadmonition}

\sphinxstepscope


\chapter{SIMULTAN Datastructure to incorporate the RFEM Data model}
\label{\detokenize{SIMULTAN_Datastructure_to_incorporate_the_RFem_Data_model:simultan-datastructure-to-incorporate-the-rfem-data-model}}\label{\detokenize{SIMULTAN_Datastructure_to_incorporate_the_RFem_Data_model:id1}}\label{\detokenize{SIMULTAN_Datastructure_to_incorporate_the_RFem_Data_model::doc}}

\section{Digital Twin and data model for structural analysis}
\label{\detokenize{SIMULTAN_Datastructure_to_incorporate_the_RFem_Data_model:digital-twin-and-data-model-for-structural-analysis}}
\sphinxAtStartPar
In \hyperref[\detokenize{SIMULTAN_Datastructure_to_incorporate_the_RFem_Data_model:struct-anal-comp}]{Fig.\@ \ref{\detokenize{SIMULTAN_Datastructure_to_incorporate_the_RFem_Data_model:struct-anal-comp}}} a simplified depiction of the digital twin and the \sphinxcode{\sphinxupquote{Structural Analysis Component}} is
given. This component is used in the SIMULTAN datamodel to store the information needed to export data with the RFEM
Plugin to enable structural analysis simulations in RFEM 6. The global structural analysis can be simplified into 3 main
containers of information, which are independent of each other. They are represented by their own compoenents in the
SIMULTAN data model, everyone of them holding sub\sphinxhyphen{}components for further detailed information.

\begin{figure}[htbp]
\centering
\capstart

\noindent\sphinxincludegraphics[height=750\sphinxpxdimen]{{schematic_components_struc}.jpg}
\caption{Simplified depiction of the representation of a Digital Twin with a \sphinxcode{\sphinxupquote{Structural Analysis Component}}.}\label{\detokenize{SIMULTAN_Datastructure_to_incorporate_the_RFem_Data_model:struct-anal-comp}}\end{figure}

\sphinxAtStartPar
The connection and dependency between some of these components is indicated in \hyperref[\detokenize{SIMULTAN_Datastructure_to_incorporate_the_RFem_Data_model:struct-anal-comp}]{Fig.\@ \ref{\detokenize{SIMULTAN_Datastructure_to_incorporate_the_RFem_Data_model:struct-anal-comp}}} by the blue
lines with arrows next to \sphinxcode{\sphinxupquote{Component with physical information}}, \sphinxcode{\sphinxupquote{Component for geometric assignment}} and
the \sphinxcode{\sphinxupquote{Structural Analysis Component}}.

\sphinxAtStartPar
In the SIMULTAN Editor these components could be represented in the ways shown in \hyperref[\detokenize{SIMULTAN_Datastructure_to_incorporate_the_RFem_Data_model:digitwin-simultan}]{Fig.\@ \ref{\detokenize{SIMULTAN_Datastructure_to_incorporate_the_RFem_Data_model:digitwin-simultan}}} and
\hyperref[\detokenize{SIMULTAN_Datastructure_to_incorporate_the_RFem_Data_model:comp-simultan}]{Fig.\@ \ref{\detokenize{SIMULTAN_Datastructure_to_incorporate_the_RFem_Data_model:comp-simultan}}}.

\begin{figure}[htbp]
\centering
\capstart

\noindent\sphinxincludegraphics[height=250\sphinxpxdimen]{{digitwin_simultan}.jpg}
\caption{Collapsed model of a digital twin for a \sphinxstyleemphasis{Tiny House} in the SIMULTAN Editor.}\label{\detokenize{SIMULTAN_Datastructure_to_incorporate_the_RFem_Data_model:digitwin-simultan}}\end{figure}

\begin{figure}[htbp]
\centering
\capstart

\noindent\sphinxincludegraphics[height=750\sphinxpxdimen]{{comp_simultan}.jpg}
\caption{Representation of the depicted components in the SIMULTAN Editor.}\label{\detokenize{SIMULTAN_Datastructure_to_incorporate_the_RFem_Data_model:comp-simultan}}\end{figure}


\section{Structural Analysis Component in depth}
\label{\detokenize{SIMULTAN_Datastructure_to_incorporate_the_RFem_Data_model:structural-analysis-component-in-depth}}
\sphinxAtStartPar
If we take a more detailed look into the \sphinxcode{\sphinxupquote{Structural Analysis Component}} we can see the references and dependencies
created to link and store information in \hyperref[\detokenize{SIMULTAN_Datastructure_to_incorporate_the_RFem_Data_model:struct-comp-detail}]{Fig.\@ \ref{\detokenize{SIMULTAN_Datastructure_to_incorporate_the_RFem_Data_model:struct-comp-detail}}}. Here the data structure with modelling
information, references and dependencies for a section are shown. In this case this section represents a steel beam
which will be exported to RFEM 6 with the plugin for a finite element analysis.

\begin{figure}[htbp]
\centering
\capstart

\noindent\sphinxincludegraphics[height=750\sphinxpxdimen]{{struct_comp_detail}.jpg}
\caption{Detailed depiction of the representation of a section in the Digital Twin.}\label{\detokenize{SIMULTAN_Datastructure_to_incorporate_the_RFem_Data_model:struct-comp-detail}}\end{figure}


\section{Data modelling decisions for data representation}
\label{\detokenize{SIMULTAN_Datastructure_to_incorporate_the_RFem_Data_model:data-modelling-decisions-for-data-representation}}
\sphinxAtStartPar
Due to the fact that in structural analysis often times the same data is used in different geometrical locations (e.g.
force attacking at multiple locations) the component storing the information for these repeatedly used data is only
created once and instances of the component are storing the differing geomtrical information. This helps to avoid
cluttering of the data model is avoided.

\sphinxAtStartPar
The instances can be accessed from the SIMULTAN Editor as shown in \hyperref[\detokenize{SIMULTAN_Datastructure_to_incorporate_the_RFem_Data_model:show-instances}]{Fig.\@ \ref{\detokenize{SIMULTAN_Datastructure_to_incorporate_the_RFem_Data_model:show-instances}}} and further information is
displayed in the opened tab, \hyperref[\detokenize{SIMULTAN_Datastructure_to_incorporate_the_RFem_Data_model:instance-window}]{Fig.\@ \ref{\detokenize{SIMULTAN_Datastructure_to_incorporate_the_RFem_Data_model:instance-window}}}.

\begin{figure}[htbp]
\centering
\capstart

\noindent\sphinxincludegraphics[height=250\sphinxpxdimen]{{show_instances}.png}
\caption{Right click to access instances of a component.}\label{\detokenize{SIMULTAN_Datastructure_to_incorporate_the_RFem_Data_model:show-instances}}\end{figure}

\begin{figure}[htbp]
\centering
\capstart

\noindent\sphinxincludegraphics[height=250\sphinxpxdimen]{{instance_window}.png}
\caption{Further information can be accessed from the opened instances tab.}\label{\detokenize{SIMULTAN_Datastructure_to_incorporate_the_RFem_Data_model:instance-window}}\end{figure}

\sphinxstepscope


\chapter{Copyright and license agreements}
\label{\detokenize{LICENSE:copyright-and-license-agreements}}\label{\detokenize{LICENSE::doc}}
\sphinxAtStartPar
MIT License

\sphinxAtStartPar
Copyright (c) 2022 RWT Plus ZT GmbH and TU Wien

\sphinxAtStartPar
Permission is hereby granted, free of charge, to any person obtaining a copy of this software and associated
documentation files (the “Software”), to deal in the Software without restriction, including without limitation the
rights to use, copy, modify, merge, publish, distribute, sublicense, and/or sell copies of the Software, and to permit
persons to whom the Software is furnished to do so, subject to the following conditions:

\sphinxAtStartPar
The above copyright notice and this permission notice shall be included in all copies or substantial portions of the
Software.

\sphinxAtStartPar
THE SOFTWARE IS PROVIDED “AS IS”, WITHOUT WARRANTY OF ANY KIND, EXPRESS OR IMPLIED, INCLUDING BUT NOT LIMITED TO THE
WARRANTIES OF MERCHANTABILITY, FITNESS FOR A PARTICULAR PURPOSE AND NONINFRINGEMENT. IN NO EVENT SHALL THE AUTHORS OR
COPYRIGHT HOLDERS BE LIABLE FOR ANY CLAIM, DAMAGES OR OTHER LIABILITY, WHETHER IN AN ACTION OF CONTRACT, TORT OR
OTHERWISE, ARISING FROM, OUT OF OR IN CONNECTION WITH THE SOFTWARE OR THE USE OR OTHER DEALINGS IN THE SOFTWARE.

\sphinxAtStartPar
Please reference this repo when including information or knowledge provided here in your work.

\sphinxstepscope


\chapter{References}
\label{\detokenize{References:references}}\label{\detokenize{References:id1}}\label{\detokenize{References::doc}}
\begin{sphinxthebibliography}{BuhlerSB}
\bibitem[Betal]{References:id4}
\sphinxAtStartPar
\sphinxstylestrong{missing year in bednarSIMULTANSimultanePlanungsumgebung}
\bibitem[BuhlerSB22]{References:id5}
\sphinxAtStartPar
Maximilian Bühler, Bernhard Steiner, and Thomas Bednar. Digital Twin applications using the SIMULTAN data model and Python. In \sphinxstyleemphasis{WBC 2022}. Melbourne, 2022.
\bibitem[PLW+19]{References:id8}
\sphinxAtStartPar
Galina Paskaleva, Thomas Lewis, Sabine Wolny, Bernhard Steiner, and Thomas Bednar. SIMULTAN as a Big\sphinxhyphen{}Open\sphinxhyphen{}Real\sphinxhyphen{}BIM Data Model \sphinxhyphen{} Proof of Concept for the Design Phase. In \sphinxstyleemphasis{CIB World Building Congress 2019 Constructing Smart Cities}, 10. 2019.
\end{sphinxthebibliography}







\renewcommand{\indexname}{Index}
\printindex
\end{document}