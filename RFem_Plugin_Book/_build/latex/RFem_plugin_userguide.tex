%% Generated by Sphinx.
\def\sphinxdocclass{jupyterBook}
\documentclass[letterpaper,10pt,english]{jupyterBook}
\ifdefined\pdfpxdimen
   \let\sphinxpxdimen\pdfpxdimen\else\newdimen\sphinxpxdimen
\fi \sphinxpxdimen=.75bp\relax
\ifdefined\pdfimageresolution
    \pdfimageresolution= \numexpr \dimexpr1in\relax/\sphinxpxdimen\relax
\fi
%% let collapsible pdf bookmarks panel have high depth per default
\PassOptionsToPackage{bookmarksdepth=5}{hyperref}
%% turn off hyperref patch of \index as sphinx.xdy xindy module takes care of
%% suitable \hyperpage mark-up, working around hyperref-xindy incompatibility
\PassOptionsToPackage{hyperindex=false}{hyperref}
%% memoir class requires extra handling
\makeatletter\@ifclassloaded{memoir}
{\ifdefined\memhyperindexfalse\memhyperindexfalse\fi}{}\makeatother

\PassOptionsToPackage{warn}{textcomp}

\catcode`^^^^00a0\active\protected\def^^^^00a0{\leavevmode\nobreak\ }
\usepackage{cmap}
\usepackage{fontspec}
\defaultfontfeatures[\rmfamily,\sffamily,\ttfamily]{}
\usepackage{amsmath,amssymb,amstext}
\usepackage{polyglossia}
\setmainlanguage{english}



\setmainfont{FreeSerif}[
  Extension      = .otf,
  UprightFont    = *,
  ItalicFont     = *Italic,
  BoldFont       = *Bold,
  BoldItalicFont = *BoldItalic
]
\setsansfont{FreeSans}[
  Extension      = .otf,
  UprightFont    = *,
  ItalicFont     = *Oblique,
  BoldFont       = *Bold,
  BoldItalicFont = *BoldOblique,
]
\setmonofont{FreeMono}[
  Extension      = .otf,
  UprightFont    = *,
  ItalicFont     = *Oblique,
  BoldFont       = *Bold,
  BoldItalicFont = *BoldOblique,
]



\usepackage[Bjarne]{fncychap}
\usepackage[,numfigreset=1,mathnumfig]{sphinx}

\fvset{fontsize=\small}
\usepackage{geometry}


% Include hyperref last.
\usepackage{hyperref}
% Fix anchor placement for figures with captions.
\usepackage{hypcap}% it must be loaded after hyperref.
% Set up styles of URL: it should be placed after hyperref.
\urlstyle{same}

\addto\captionsenglish{\renewcommand{\contentsname}{Inhalt}}

\usepackage{sphinxmessages}



        % Start of preamble defined in sphinx-jupyterbook-latex %
         \usepackage[Latin,Greek]{ucharclasses}
        \usepackage{unicode-math}
        % fixing title of the toc
        \addto\captionsenglish{\renewcommand{\contentsname}{Contents}}
        \hypersetup{
            pdfencoding=auto,
            psdextra
        }
        % End of preamble defined in sphinx-jupyterbook-latex %
        

\title{RFEM-Plugin for SIMULTAN}
\date{Aug 25, 2022}
\release{}
\author{Andreas Sarkany and Zsombor Jarosi}
\newcommand{\sphinxlogo}{\vbox{}}
\renewcommand{\releasename}{}
\makeindex
\begin{document}

\pagestyle{empty}
\sphinxmaketitle
\pagestyle{plain}
\sphinxtableofcontents
\pagestyle{normal}
\phantomsection\label{\detokenize{intro::doc}}


\sphinxAtStartPar
This user guide should help navigate the RFEM\sphinxhyphen{}Plugin which enables structural analysis with the SIMULTAN datamodel based
on the finite element software RFEM 6 by Dlubal.
\begin{itemize}
\item {} 
\sphinxAtStartPar
Inhalt

\begin{itemize}
\item {} 
\sphinxAtStartPar
{\hyperref[\detokenize{Introduction::doc}]{\sphinxcrossref{Introduction}}}

\item {} 
\sphinxAtStartPar
{\hyperref[\detokenize{Getting_started_with_the_RFem_Plugin::doc}]{\sphinxcrossref{Getting started with the RFEM Plugin}}}

\item {} 
\sphinxAtStartPar
{\hyperref[\detokenize{Setting_up_a_problem::doc}]{\sphinxcrossref{Setting up a problem}}}

\item {} 
\sphinxAtStartPar
{\hyperref[\detokenize{Running_a_simulation::doc}]{\sphinxcrossref{Running a simulation}}}

\item {} 
\sphinxAtStartPar
{\hyperref[\detokenize{Results_of_the_simulation::doc}]{\sphinxcrossref{Results of the simulation}}}

\item {} 
\sphinxAtStartPar
{\hyperref[\detokenize{Connecting_an_existing_model_to_the_RFem_Plugin::doc}]{\sphinxcrossref{Connecting an existing model to the RFEM Plugin}}}

\item {} 
\sphinxAtStartPar
{\hyperref[\detokenize{SIMULTAN_Datastructure_to_incorporate_the_RFem_Data_model::doc}]{\sphinxcrossref{SIMULTAN Datastructure to incorporate the RFem Data model}}}

\item {} 
\sphinxAtStartPar
{\hyperref[\detokenize{LICENSE::doc}]{\sphinxcrossref{Copyright and license agreements}}}

\item {} 
\sphinxAtStartPar
{\hyperref[\detokenize{References::doc}]{\sphinxcrossref{Literatur}}}

\end{itemize}
\end{itemize}

\sphinxstepscope


\chapter{Introduction}
\label{\detokenize{Introduction:introduction}}\label{\detokenize{Introduction:id1}}\label{\detokenize{Introduction::doc}}

\section{What is the RFEM\sphinxhyphen{}Plugin?}
\label{\detokenize{Introduction:what-is-the-rfem-plugin}}
\sphinxAtStartPar
This is a {\hyperref[\detokenize{intro:intro}]{\sphinxcrossref{\DUrole{std,std-ref}{reference}}}}.


\section{Functionality covered by the RFEM\sphinxhyphen{}Plugin}
\label{\detokenize{Introduction:functionality-covered-by-the-rfem-plugin}}

\section{Project}
\label{\detokenize{Introduction:project}}

\section{Authors}
\label{\detokenize{Introduction:authors}}

\section{Getting help}
\label{\detokenize{Introduction:getting-help}}
\sphinxstepscope


\chapter{Getting started with the RFEM Plugin}
\label{\detokenize{Getting_started_with_the_RFem_Plugin:getting-started-with-the-rfem-plugin}}\label{\detokenize{Getting_started_with_the_RFem_Plugin:id1}}\label{\detokenize{Getting_started_with_the_RFem_Plugin::doc}}

\section{Installation}
\label{\detokenize{Getting_started_with_the_RFem_Plugin:installation}}
\sphinxstepscope


\chapter{Setting up a problem}
\label{\detokenize{Setting_up_a_problem:setting-up-a-problem}}\label{\detokenize{Setting_up_a_problem:id1}}\label{\detokenize{Setting_up_a_problem::doc}}

\section{Structural analysis component}
\label{\detokenize{Setting_up_a_problem:structural-analysis-component}}
\sphinxAtStartPar
Rough overview with some figures


\section{User Interface}
\label{\detokenize{Setting_up_a_problem:user-interface}}
\sphinxstepscope


\chapter{Running a simulation}
\label{\detokenize{Running_a_simulation:running-a-simulation}}\label{\detokenize{Running_a_simulation:id1}}\label{\detokenize{Running_a_simulation::doc}}

\section{Exporting the SIMULTAN\sphinxhyphen{}model to RFEM}
\label{\detokenize{Running_a_simulation:exporting-the-simultan-model-to-rfem}}

\section{Importing into RFEM}
\label{\detokenize{Running_a_simulation:importing-into-rfem}}

\section{Running a simulation in RFEM}
\label{\detokenize{Running_a_simulation:running-a-simulation-in-rfem}}
\sphinxstepscope


\chapter{Results of the simulation}
\label{\detokenize{Results_of_the_simulation:results-of-the-simulation}}\label{\detokenize{Results_of_the_simulation:id1}}\label{\detokenize{Results_of_the_simulation::doc}}

\section{Visual results}
\label{\detokenize{Results_of_the_simulation:visual-results}}

\section{Additional information}
\label{\detokenize{Results_of_the_simulation:additional-information}}
\sphinxstepscope


\chapter{Connecting an existing model to the RFEM Plugin}
\label{\detokenize{Connecting_an_existing_model_to_the_RFem_Plugin:connecting-an-existing-model-to-the-rfem-plugin}}\label{\detokenize{Connecting_an_existing_model_to_the_RFem_Plugin:id1}}\label{\detokenize{Connecting_an_existing_model_to_the_RFem_Plugin::doc}}

\section{How and what to add to the data model}
\label{\detokenize{Connecting_an_existing_model_to_the_RFem_Plugin:how-and-what-to-add-to-the-data-model}}

\section{Creating a linked geometry}
\label{\detokenize{Connecting_an_existing_model_to_the_RFem_Plugin:creating-a-linked-geometry}}

\section{Creating the needed components}
\label{\detokenize{Connecting_an_existing_model_to_the_RFem_Plugin:creating-the-needed-components}}
\sphinxAtStartPar
Although it is possible to recreate the needed data structure for the structural analysis component in the
SIMULTAN\sphinxhyphen{}Editor manually, it is strongly recommended to use the User\sphinxhyphen{}Interface of the plugin and apply changes when they
would be necessary.

\sphinxstepscope


\chapter{SIMULTAN Datastructure to incorporate the RFem Data model}
\label{\detokenize{SIMULTAN_Datastructure_to_incorporate_the_RFem_Data_model:simultan-datastructure-to-incorporate-the-rfem-data-model}}\label{\detokenize{SIMULTAN_Datastructure_to_incorporate_the_RFem_Data_model:id1}}\label{\detokenize{SIMULTAN_Datastructure_to_incorporate_the_RFem_Data_model::doc}}
\sphinxstepscope


\chapter{Copyright and license agreements}
\label{\detokenize{LICENSE:copyright-and-license-agreements}}\label{\detokenize{LICENSE::doc}}
\sphinxAtStartPar
MIT License

\sphinxAtStartPar
Copyright (c) 2022 Andreas Sarkany, Zsombor Jarosi

\sphinxAtStartPar
Permission is hereby granted, free of charge, to any person obtaining a copy of this software and associated
documentation files (the “Software”), to deal in the Software without restriction, including without limitation the
rights to use, copy, modify, merge, publish, distribute, sublicense, and/or sell copies of the Software, and to permit
persons to whom the Software is furnished to do so, subject to the following conditions:

\sphinxAtStartPar
The above copyright notice and this permission notice shall be included in all copies or substantial portions of the
Software.

\sphinxAtStartPar
THE SOFTWARE IS PROVIDED “AS IS”, WITHOUT WARRANTY OF ANY KIND, EXPRESS OR IMPLIED, INCLUDING BUT NOT LIMITED TO THE
WARRANTIES OF MERCHANTABILITY, FITNESS FOR A PARTICULAR PURPOSE AND NONINFRINGEMENT. IN NO EVENT SHALL THE AUTHORS OR
COPYRIGHT HOLDERS BE LIABLE FOR ANY CLAIM, DAMAGES OR OTHER LIABILITY, WHETHER IN AN ACTION OF CONTRACT, TORT OR
OTHERWISE, ARISING FROM, OUT OF OR IN CONNECTION WITH THE SOFTWARE OR THE USE OR OTHER DEALINGS IN THE SOFTWARE.

\sphinxAtStartPar
Please reference this repo when including information or knowledge provided here in your work

\sphinxstepscope


\chapter{Literatur}
\label{\detokenize{References:literatur}}\label{\detokenize{References:references}}\label{\detokenize{References::doc}}\phantomsection\label{\detokenize{References:id1}}






\renewcommand{\indexname}{Index}
\printindex
\end{document}